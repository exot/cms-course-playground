\documentclass{beamer}

\usepackage[utf8]{inputenc}
\usepackage[T1]{fontenc}
\usepackage[ngerman]{babel}

\usetheme{Berlin}

% wichtig: üblicherweise ungenutzte Navigationssymbole ausblenden
\setbeamertemplate{navigation symbols}{}

\usepackage{csquotes}
\usepackage{color,graphicx}

\usepackage{multimedia}
\usepackage{tikz}
\usepackage{tikzducks}

\begin{document}

\title{Eine kleine Präsentation}
\author{Johnna Doe}
\date{1970-01-01}

\begin{frame}
  \maketitle
\end{frame}

\begin{frame}
  \frametitle{Agenda}
  \tableofcontents{}
\end{frame}

\section{Was ist eigentlich \texttt{beamer}?}

\begin{frame}
  \frametitle{Aktuell}
  \tableofcontents[currentsection]{}
\end{frame}

\begin{frame}
  \frametitle{\texttt{beamer}}

  \onslide<+->

  Mit \texttt{beamer} kann man direkt in \LaTeX{} Präsentationen setzen, mit
  allem drum und dran.

  \onslide<+->

  \bigskip{}

  Auch Einblendungen sind möglich.

  \onslide<+->

  \bigskip{}

  Und das ist nicht mal schwer!

\end{frame}

\begin{frame}
  \frametitle{und sonst so}

  \onslide<+->

  \begin{columns}
    \begin{column}{0.45\linewidth}
      \begin{itemize}
      \item<+-> Mehrspaltig geht auch
      \item<+-> und mit Aufzählungen!
      \item<+-> mehr oder weniger eben alles, was \LaTeX{} kann
      \end{itemize}
    \end{column}

    \begin{column}{0.45\linewidth}
      \onslide<+->

      Und Bilder gehen auch!

      \begin{center}
        \includegraphics[width=\linewidth]{img-lion.pdf}
      \end{center}

    \end{column}
  \end{columns}

\end{frame}

\begin{frame}
  \frametitle{Tabellen}
  \begin{tabular}{|c|c|}\hline
   A & B \\ \hline
   a11 & a12 \\ \hline
   a21 & a22 \\ \hline
 \end{tabular}
\end{frame}

\begin{frame}[fragile]
  \frametitle{Doku?}

  \onslide<+->

  \begin{itemize}
  \item \verb!texdoc beameruserguide!
  \item \url{https://tex.stackexchange.com/}
  \item \url{https://www.dante.de/tex/TeXAnfaenger.html}
  \item ganz allgemein DAS INTERNET!
  \end{itemize}

  \onslide<+->

  \begin{block}{oder einfach}
    \begin{center}
      Leute fragen \\\tikz{\duck}
    \end{center}

  \end{block}

\end{frame}

\section{Was sollte man mit \texttt{beamer} nicht machen}

\begin{frame}
  \frametitle{Aktuell}
  \tableofcontents[currentsection]{}
\end{frame}

\begin{frame}
  \frametitle{Don'ts}

  \begin{columns}
    \begin{column}{0.45\linewidth}

  \begin{itemize}
  \item<+-> Videos einbetten
  \item<+-> \enquote{lustige} Einblendungen (das geht zwar, wird aber nicht von
    allen PDF-Viewern unterstützt)
  \item<+-> alles, was man sowieso nicht in Präsentationen machen sollte \ldots
  \end{itemize}

    \end{column}
    \begin{column}{0.45\linewidth}
\IfFileExists{ThisLandIsMine-byNinaPaley.mkv}{
	\movie[width=50mm,height=28mm,poster,externalviewer]{%
		\IfFileExists{ThisLandIsMine-byNinaPaley.jpeg}{
			\includegraphics[width=50mm]{ThisLandIsMine-byNinaPaley.jpeg}
		}{
			This land is mine. by Nina Paley
		}
	}{ThisLandIsMine-byNinaPaley.mkv}
}{
	\url{https://www.youtube.com/watch?v=4pKMV6e5kEo}
}
	\end{column}
  \end{columns}

\end{frame}

\begin{frame}[fragile]
  \frametitle{WTF?}

  \onslide<+->

  \begin{multline*}
    \int \limits _{0}^{\infty }{\frac {1+{\dfrac {x^{2}}{(b+1)^{2}}}}{1+{\dfrac
          {x^{2}}{a^{2}}}}}\times {\frac {1+{\dfrac
          {x^{2}}{(b+2)^{2}}}}{1+{\dfrac {x^{2}}{(a+1)^{2}}}}}\times \cdots \,dx \\
    = {\frac {\sqrt {\pi }}{2}}\times {\frac {\Gamma \left(a+{\frac
            {1}{2}}\right)\Gamma (b+1)\Gamma (b-a+1)}{\Gamma (a)\Gamma
        \left(b+{\frac {1}{2}}\right)\Gamma \left(b-a+{\frac {1}{2}}\right)}}.
  \end{multline*}

  \bigskip

  {\tiny \url{https://en.wikipedia.org/wiki/Srinivasa_Ramanujan}}

  \onslide<+->

  \begin{tikzpicture}[remember picture, overlay]
    \node[rotate=40] at (current page.center) {{\Huge\color{red}\bfseries ALLES MIST!}};
  \end{tikzpicture}

\end{frame}

\section{Fazit}

\begin{frame}
  \frametitle{Aktuell}
  \tableofcontents[currentsection]{}
\end{frame}

\begin{frame}
  \frametitle{Und was soll das alles?}

  \onslide<+->

  \begin{center}
    \LaTeX{}-\texttt{beamer} fetzt! \\
    \onslide<+->
    \dots und Powerpoint ist doof \dots
  \end{center}

\end{frame}

\end{document}
